\documentclass{article}
\newcommand\authormail[1]{\footnote{\textless\url{#1}\textgreater}}
\ifdefined\HCode
  \renewcommand\authormail[1]{\space\textless\Link[#1]{}{}#1\EndLink\textgreater}
\fi

\usepackage{longtable}
\usepackage{tabularx}
\newenvironment{changelog}{\longtable{@{} l p{30em}}}{\endlongtable}
\newcommand\change[2]{#1 & #2\\}

\author{Michal Hoftich\authormail{michal.h21@gmail.com}}
\ifdefined\gitdate\else
  \def\gitdate{\today}
  \def\version{devel}
\fi
\title{\texttt{Rmodepdf} -- convert web pages to PDF using \LaTeX}
\date{Version \version, \gitdate}
\usepackage{upquote}
\usepackage{microtype}
\usepackage{hyperref}
\usepackage{fontspec}
\usepackage{linebreaker}
\setmainfont{Linux Libertine O}
\usepackage{luacode}
\begin{document}
\maketitle
\begin{tabular}{l l}
  Homepage: &\url{https://www.kodymirus.cz/texblend/}\\
  Issue tracker:&  \url{https://github.com/michal-h21/texblend}
\end{tabular}
\tableofcontents

\section{Introduction}

This utility converts text content of web pages to PDF using \LaTeX. The text
content is extracted using \textit{rdrview}\footnote{\url{https://github.com/eafer/rdrview}}, utility that provides a port of Firefox's reader
view functionality. This means that it strips away clutter like buttons, ads, background images, and videos, leaving only the article text. 

It doesn't support any CSS or JavaScript, only plain HTML.
The main purpose is to create version of longer articles suitable for reading
on e-readers, tablets and phones. Another possible usage is for printing of web pages.


\section{Usage}

The basic usage is following:

\begin{verbatim}
$ rmodepdf <url>
\end{verbatim}

If the compilation goes well, Rmodepdf should print a message like:

\begin{verbatim}
[STATUS]  rmodepdf: File saved as: Page_Title.pdf
\end{verbatim}

File name of the PDF name is based on the web page title. You can choose
a different filename using the \verb|-o| option:

\begin{verbatim}
$ rmodepdf -o sample <url>
\end{verbatim}

You can also compile several web pages at once, Rmodepdf will convert all URLs
passed as argument as one PDF, with the filename based on the first page's title:

\begin{verbatim}
$ rmodepdf <url1> <url2> <url3>
\end{verbatim}

Instead of URLs, you can also pass filenames of local files or pass the HTML code from the 
standard input with the \verb|-| option:

\begin{verbatim}
$ rmodepdf - < localfile.html
\end{verbatim}





\section{Command Line Options}


\begin{verbatim}
-b,--baseurl     (default "")      Base URL used when the HTML content is read
                                   from the standard input
-c,--configfile  (default "")      Filename of Lua configuration file
-h,--help                          Print help message
-i,--imgdir      (default "")      Download images and save them to the
                                   specified directory
-l,--loglevel    (default status)  Set log level
                                   possible values: debug, info, status,
                                   warning, error, fatal
-n,--noimages                      Don't download images
-N,--nomathjax                     Don't process LaTeX commands in the HTML
                                   document
-t,--template    (default "")      LaTeX template
-o,--output      (default "")      Output file name
-p,--pageformat  (default ebook)   Page format 
-R,--nordrview                     Don't use rdrview to get the clean contents
                                   from the web pages
-s,--pagestyle   (default empty)   \pagestyle for the document
-p,--print                         Print the converted LaTeX source
-v,--version                       Print version
<url>            (string)
\end{verbatim}



\subsection{Image Handling}

By default, Rmodepdf downloads all images and saves them as temporary files which are removed
after each run. If you want to reuse these images, use the \verb|--imgdir| option. It expects
an existing directory where images should be saved.

\begin{verbatim}
$ rmodepdf -i img <url>
\end{verbatim}

If you read HTML content from the standard input, you can use the \verb|--baseurl| option to  
point to the adress where images should be looked up.

The \verb|--noimages| option on the other hand will disable downloading of images.

\subsection{MathJax Support}

Rmodepdf expects web pages to use MathJax or KaTeX libraries, which enables \LaTeX\ syntax for math 
in the HTML content. In some cases, this can lead to errors. For example if  \LaTeX\ commands are 
displayed in the HTML code outside of \verb|<code>| or \verb|<pre>| elements. The \verb|--nomathjax| option
will disable passing of \LaTeX\ commands to the resulting document.


\subsection{Document Settings}

\section{Configuration}

\section{Scripting}

\section{License}

Permission is granted to copy, distribute and/or modify this software
under the terms of the \LaTeX\ Project Public License, version 1.3.

\section{Changelog}
\begin{changelog}
  \change{2024-04-09}{Added \texttt{--nordrview} option}
  \change{}{Basic metadata parsing if rdrview is not available or is disabled}
  \change{2024-04-08}{ChangeLog start}
\end{changelog}

\end{document}

