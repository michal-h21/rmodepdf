\documentclass{article}
\newcommand\authormail[1]{\footnote{\textless\url{#1}\textgreater}}
\ifdefined\HCode
  \renewcommand\authormail[1]{\space\textless\Link[#1]{}{}#1\EndLink\textgreater}
\fi

\usepackage{longtable}
\usepackage{tabularx}
\newenvironment{changelog}{\longtable{@{} l p{30em}}}{\endlongtable}
\newcommand\change[2]{#1 & #2\\}

\author{Michal Hoftich\authormail{michal.h21@gmail.com}}
\ifdefined\gitdate\else
  \def\gitdate{\today}
  \def\version{devel}
\fi
\title{\texttt{Rmodepdf} -- convert web pages to PDF using \LaTeX}
\date{Version \version, \gitdate}
\usepackage{upquote}
\usepackage{microtype}
\usepackage{hyperref}
\usepackage{fontspec}
\usepackage{linebreaker}
\setmainfont{Linux Libertine O}
\usepackage{luacode}
\begin{document}
\maketitle
\begin{tabular}{l l}
  Homepage: &\url{https://www.kodymirus.cz/texblend/}\\
  Issue tracker:&  \url{https://github.com/michal-h21/texblend}
\end{tabular}
\tableofcontents

\section{Introduction}

This utility converts text content of web pages to PDF using \LaTeX. The text
content is extracted using \textit{rdrview}\footnote{\url{https://github.com/eafer/rdrview}}, utility that provides a port of Firefox's reader
view functionality. This means that it strips away clutter like buttons, ads, background images, and videos, leaving only the article text. 

It doesn't support any CSS or JavaScript, only plain HTML.
The main purpose is to create version of longer articles suitable for reading
on e-readers, tablets and phones. Another possible usage is for printing of web pages.


\section{Usage}

The basic usage is following:

\begin{verbatim}
$ rmodepdf <url>
\end{verbatim}

If the compilation goes well, Rmodepdf should print a message like:

\begin{verbatim}
[STATUS]  rmodepdf: File saved as: Page_Title.pdf
\end{verbatim}

File name of the PDF name is based on the web page title. You can choose
a different filename using the \verb|-o| option:

\begin{verbatim}
$ rmodepdf -o sample <url>
\end{verbatim}

You can also compile several web pages at once, Rmodepdf will convert all URLs
passed as argument as one PDF, with the filename based on the first page's title:

\begin{verbatim}
$ rmodepdf <url1> <url2> <url3>
\end{verbatim}

Instead of URLs, you can also pass filenames of local files or pass the HTML code from the 
standard input with the \verb|-| option:

\begin{verbatim}
$ rmodepdf - < localfile.html
\end{verbatim}





\section{Command Line Options}


\begin{verbatim}
-b,--baseurl       (default "")      Base URL used when the HTML content is read
                                     from the standard input
-c,--configfile    (default "")      Filename of Lua configuration file
-h,--help                            Print help message
-H,--nohyperlinks                    Don't create special elements for internal hyperlinks 
-i,--imgdir        (default "")      Download images and save them to the
                                     specified directory
-l,--loglevel      (default status)  Set log level
                                     possible values: debug, info, status,
                                     warning, error, fatal
-n,--noimages                        Don't download images
-N,--nomathjax                       Don't process LaTeX commands in the HTML
                                     document
-t,--template      (default "")      LaTeX template
-o,--output        (default "")      Output file name
-p,--pageformat    (default ebook)   Page format 
-R,--nordrview                       Don't use rdrview to get the clean contents
                                     from the web pages
-s,--pagestyle     (default empty)   \pagestyle for the document
-p,--print                           Print the converted LaTeX source
-v,--version                         Print version
<url>              (string)
\end{verbatim}



\subsection{Image Handling}

By default, Rmodepdf downloads all images and saves them as temporary files which are removed
after each run. If you want to reuse these images, use the \verb|--imgdir| option. It expects
an existing directory where images should be saved.

\begin{verbatim}
$ rmodepdf -i img <url>
\end{verbatim}

If you read HTML content from the standard input, you can use the \verb|--baseurl| option to  
point to the adress where images should be looked up.

The \verb|--noimages| option on the other hand will disable downloading of images.

\subsection{MathJax Support}

Rmodepdf expects web pages to use MathJax or KaTeX libraries, which enables \LaTeX\ syntax for math 
in the HTML content. In some cases, this can lead to errors. For example if  \LaTeX\ commands are 
displayed in the HTML code outside of \verb|<code>| or \verb|<pre>| elements. The \verb|--nomathjax| option
will disable passing of \LaTeX\ commands to the resulting document.


\subsection{Page format}




\section{Configuration}


\begin{verbatim}
add_to_config {
  img_convert = { 
    -- modify the command used for conversion of svg images to 
    -- a format suitable for LuaLaTeX
    svg = "cairosvg -o ${dest} -", 
  },
  html_latex = { -- support for LaTeX math in webpages that use MathJax or KaTeX
    ignored = {"pre", "code"}, -- html elements which shouldn't be processed for LaTeX commands
    allowed_commands = {"ref", "pageref", "cleveref", "nameref"}
  },
}
\end{verbatim}]


\begin{verbatim}
function post_process()
  -- set French as a main document language
  table.insert(config.document.languages, "french")
end
\end{verbatim}


\subsection{The \texttt{document} table}

\begin{description}

  \item[preamble\_extras] -- additional code to be inserted at the end of the
    document preamble. For example font settings, extra packages, etc.
  \item[geometry] -- string with page dimensions in format suitable for the Geometry package. 
  \item[pagestyle] -- document page style.
  \item[languages] -- list of languages used by the processed pages. This is populated during page processing.


\end{description}

\subsection{The \texttt{pages} table}

The \texttt{config.pages} table contains list of all processed HTML documents and their metadata.
Each item in the list contains the following properties:

\begin{description}
  \item[language] -- language of the document.
  \item[content] -- result of HTML to \TeX\ conversion.
  \item[author] -- document author.
  \item[title] -- document title.
  \item[url] -- document URL.
\end{description}


\section{\LaTeX\ Templates}


\subsection{Syntax}

\begin{description}
  \item[Variable Printing] \verb|@{variablename}|: Variables are contained in the \verb|config| table. Using a dot, properties of sub-tables can also be printed. For example, \texttt{@\{document.\allowbreak preamble\_extras\}} prints the \texttt{config.\allowbreak document.preamble\_extras} variable.
  
  \item[Loops] \verb|_{variablename}loop code/{separator}|: Variables used must be arrays. For example, \verb|document.languages| contains the languages of all translated documents in a format suitable for the Babel package, or \verb|pages|, which contains all converted documents. In the loop code, variables of the currently processed array are available. If the array contains only strings, the placeholder \verb|%s| can be used, as with \verb|document.languages|. If the current object is a table, its fields can be accessed directly using \verb|@{variablename}|.
  
  \item[Conditions] \verb|?{variablename}{true}{false}|: Used to insert elements like the title and author, which may not be present on all pages.
\end{description}

\begin{verbatim}
\documentclass{article}
\usepackage{linebreaker,responsive}
\usepackage[_{document.languages}%s/{,}] {babel}
\usepackage[@{document.geometry}]{geometry}
\pagestyle{@{document.pagestyle}}
@{document.preamble_extras}
\begin{document}
_{pages}
\selectlanguage{@{language}}
?{title}{Title: @{title}}\par}{}
?{author}{Author: @{author}\par}{}
\href{@{url}}{@{url}}\par
@{content}
/{\clearpage}
\end{document}
\end{verbatim}

Note that when processing an array, we must distinguish whether it contains
strings or tables. Strings are displayed using \verb|%s|. If it is a table, its
elements become active variables and can be displayed using
\verb|@{variablename}|. You can see the difference in processing the array
\verb|document.languages|, which contains languages as strings, and
\verb|pages|, which contains tables with metadata from processed pages.





\subsubsection{Variable expansion}

\begin{verbatim}
\end{verbatim}

\subsection{Required packages}


The default templates used for conversion from HTML to \LaTeX\ utilize some
commands that are not available in pure \LaTeX. If you are creating your own
template, it is necessary to use the following packages in it to avoid
compilation errors.

\begin{description}
  \item[cals] -- table support
  \item[csquotes] -- multilinugal support for in-text quotes
  \item[adjustbox] -- automatic resizing of images, to fit into page dimensions
  \item[responsive] -- set font sizes to fit into page dimensions
\end{description}

\section{Scripting}

The configuration script is executed before the actual conversion, so it cannot
directly influence the conversion process. However, we can define several
callback functions that allow us to affect the conversion. These functions are
as follows:

\begin{description}
  \item[\texttt{preprocess\_content}] -- modify string with the raw HTML before
    readability and DOM parsing.
  \item[\texttt{preprocess\_dom}] -- modify DOM object before fetchching of images
    or handling of MathJax.
  \item[\texttt{postprocess\_dom}] -- modify DOM after all processing by Rmodepdf.
  \item[\texttt{postprocess}] -- late post-processing of the config table.
\end{description}


\section{License}

Permission is granted to copy, distribute and/or modify this software
under the terms of the \LaTeX\ Project Public License, version 1.3.

\section{Changelog}
\begin{changelog}
  \change{2024-07-25}{Use special elements for internal links in the document. This can be dissabled with the \texttt{--nohyperlinks} option.}
  \change{2024-07-22}{Changed the MathJax handling code. It now adds a special element only for the math itself, not for the surrounding text.}
  \change{2024-06-13}{Added new hook, \texttt{preprocess\_content()}, for modyfying of the raw HTML string}
  \change{}{Clenup of some unused code}
  \change{}{Added \texttt{add\_to\_config()} function, for modyfying of the configuraton table.}
  \change{2024-06-12}{Provided new templating mechanism that doesn't depend on LuaXML templates}
  \change{2024-04-09}{Added \texttt{-\/-nordrview} option}
  \change{}{Basic metadata parsing if rdrview is not available or is disabled}
  \change{2024-04-08}{ChangeLog start}
\end{changelog}

\end{document}

